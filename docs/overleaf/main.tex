\
\documentclass[11pt]{article}
\usepackage[a4paper,margin=1in]{geometry}
\usepackage{amsmath,amssymb}
\usepackage{siunitx}
\usepackage{booktabs}
\usepackage{graphicx}
\usepackage[hidelinks]{hyperref}
\usepackage{cleveref}
\usepackage{microtype}

\title{UK Debt Sustainability: Baseline Replication and Risk Scenarios}
\author{Giorgos Papachristodoulou \and Alex Michaelides}
\date{\today}

\begin{document}
\maketitle

\begin{abstract}
We replicate the OBR baseline and stress-test UK debt sustainability using deterministic and stochastic scenarios. All results are fully reproducible from code and up-to-date public data.
\end{abstract}

\section{Introduction and Contribution}
One-paragraph context; clear statement of contribution and computational approach.

\section{Data and Baseline Replication}
Document OBR/ONS/DMO/BoE sources; show reconciliation table. Figures auto-generated from \texttt{figures/auto}.

\section{Methods}
Debt arithmetic, maturity mapping, term-structure transmission, Monte-Carlo design.

\section{Results: Deterministic Scenarios}
Clean figures, no fluff; policy-relevant metrics (debt/GDP, interest/revenue, financing need).

\section{Results: Stochastic Distributions}
Fan charts; probability of threshold breaches; stabilising primary balances.

\section{Policy Takeaways}
Quantified targets; credibility and risk considerations.

\bibliographystyle{abbrv}
\bibliography{refs}
\end{document}
